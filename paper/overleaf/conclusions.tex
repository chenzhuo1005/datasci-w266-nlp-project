\section{Conclusions}

This study has systematically explored the enhancement of Open-Domain Question Answering (ODQA) capabilities through the integration of retrieval-augmented generation with large language models (LLMs). The deployment of an external knowledge augmentation system utilizing approximately 1.4 million Question-Answer pairs from the \emph{PAQ} dataset has demonstrated significant advancements in model performance across various metrics.

Our findings reveal that the augmentation of traditional LLMs like \emph{Flan-T5-base} and \emph{Flan-T5-large} with Retrieval-Augmented Generation (RAG) not only substantially improves their \emph{Exact Match (EM)}, \emph{F1}, and \emph{Rouge scores} but also addresses inherent limitations due to the static nature of their trained parameters. For instance, \emph{Flan-T5-base}, enhanced with RAG, exhibited a remarkable improvement, with \emph{EM scores} increasing by over 23 percentage points, highlighting the effectiveness of leveraging external, dynamically retrieved knowledge.

Moreover, the comparison between models of different capacities, from the 220 million parameter \emph{Flan-T5-base} to the 13 billion parameter \emph{Llama-2-13B-chat-GGUF}, has underscored the scale's impact on knowledge comprehension and retrieval capabilities. Interestingly, while larger models inherently performed better in baseline settings, the marginal gains provided by RAG were proportionally greater in smaller models, suggesting a significant enhancement in their ability to contextualize and reason with external information.

The \emph{Llama-2-13B-chat-GGUF} model, with its vast parameter count, initially displayed robust performance without external aids. However, the application of RAG still offered slight improvements, emphasizing that even the most powerful models can benefit from access to expanded knowledge bases. This is particularly evident in the nuanced domain of question answering, where contextual relevance and up-to-date information are crucial.

Furthermore, the nuanced differences in performance increments between models also point to the varying effects of knowledge augmentation depending on the model architecture and initial training data. For example, \emph{GPT-3.5-Turbo}, despite its sophisticated capabilities, showcased a unique dependency on the RAG for optimizing its outputs, particularly when faced with questions tied closely to recent data.

In conclusion, this study not only reinforces the utility of integrating retrieval-augmented techniques into existing LLM frameworks but also highlights the critical need for continuous adaptation and enhancement of these models to maintain relevance over time. As the field of NLP progresses, strategies that merge static knowledge with dynamic retrieval mechanisms will be pivotal in addressing the evolving complexities of language understanding and generation. The promising results from various configurations of model and augmentation setups offer a clear pathway for future research focused on refining these integrations, potentially setting new benchmarks in the ODQA landscape.