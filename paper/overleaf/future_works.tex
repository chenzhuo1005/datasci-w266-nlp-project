\section{Future Works}

\paragraph{Exploring Advanced Embedding Models} To improve the effectiveness of our knowledge-building process, exploring various embedding models could be beneficial. For instance, evaluating the performance of models such as \emph{sentence-transformers/paraphrase-MiniLM-L6-v2}~\citep{reimers-2019-sentence-bert} or \emph{openai/text-embedding-3-small}~\citep{openaiembeddings} may enhance the quality of knowledge retrieval. Utilizing more sophisticated embedding techniques can potentially lead to significant improvements in how information is indexed and accessed.

\paragraph{Diversifying Data Sources} We could also enhance the diversity of our data sources or combine multiple datasets to enrich our knowledge base. Potential resources include NaturalQuestions~\citep[NQ,][]{NaturalQuestions}, TriviaQA~\citep[TQA,][]{TriviaQA}, and WebQuestions~\citep{WebQuestions}. By integrating various datasets, we can create a more robust and comprehensive repository of information, which may improve the performance and versatility of our retrieval systems.

\paragraph{Optimizing Data Precision} Currently, our knowledge embeddings are generated with 32-bit precision. However, studies indicate that using 8-bit embeddings could maintain performance levels while reducing the 75\% GPU memory and knowledge database size by 4 times~\citep{dettmers2023spqr, dettmers2023qlora, dettmers2022llm, DBLP:journals/corr/abs-2110-02861}. This reduction would be highly beneficial for the RAG framework, particularly as it addresses one of its key challenges: managing and accessing large volumes of data. Additionally, this approach would help in minimizing the computational resources required.

\paragraph{Rethinking Retrieval Mechanisms} Finally, the conventional method in RAG for retrieving knowledge involves searching for the top-k highest scored embedding vectors based on cosine similarity, MIPS implementations such as \texttt{faiss}~\citep{faiss} enable searching across millions of vectors in milliseconds on a CPU. However, the constructed knowledge base might exhibit dependencies, and answering complex questions may require integrating multiple knowledge segments. Rather than relying solely on independent vector similarity, incorporating conditional probability or exploring sequence-to-sequence searches and attention mechanisms~\citep{vaswani2023attention} in the knowledge retrieval process could provide a more nuanced and context-aware approach. This adjustment might significantly enhance the capability of the RAG system to generate more accurate and contextually relevant responses.